%%%% Paramétrage du TD %%%%
\def\xxactivite{ \ifprof {Application 2 -- Corrigé } \else  Application 2\fi} % \normalsize \vspace{-.4cm}
\def\xxauteur{\textsl{Xavier Pessoles}}


%\def\xxnumchapitre{Chapitre 1 \vspace{.2cm}}
%\def\xxchapitre{\hspace{.12cm} Introduction à la dynamique du solide indéformable}

\def\xxcompetences{%
\textsl{%
\textbf{Savoirs et compétences :}\\
\begin{itemize}[label=\ding{112},font=\color{bleuxp}] 
\item Mod2.C18.SF1 : Déterminer l’énergie cinétique d’un solide, ou d’un ensemble de solides, dans son mouvement par rapport à un autre solide.
\item Res1.C1.SF1 : Proposer une démarche permettant la détermination de la loi de mouvement.
\end{itemize}
}}
\def\xxtitreexo{Axe numérique}
\def\xxsourceexo{\hspace{.2cm}}%\footnotesize{Concours Commun Mines Ponts 2016}}

%\def\xxauteur{\textsl{Xavier Pessoles}}

\def\xxfigures{
\includegraphics[width=.6\linewidth]{fig_11}
}%figues de la page de garde


\iflivret
\input{\repRel/Style/pagegarde_TD}
\else
\input{../../style/new_pagegarde}
\fi
\setlength{\columnseprule}{.1pt}

\pagestyle{fancy}
\thispagestyle{plain}

\ifprof
\vspace{5.5cm}
\else
\vspace{5.5cm}
\fi

\def\columnseprulecolor{\color{bleuxp}}
\setlength{\columnseprule}{0.4pt} 
\setcounter{numques}{0}
%%%%%%%%%%%%%%%%%%%%%%%

%\ifprof
%\else
\begin{multicols}{2}
%\fi
%\section*{Exercice 1 -- Lois de Kirchoff}
%\ifprof
%\else
%\fi
%
%\subparagraph*{}
%\textit{Sur le circuit suivant, déterminer les courants dans chacune des branches et la tension aux bornes de tous les dipôles en fonction de $E$ et des différentes résistances $R_i$.}
%\begin{center}
%\includegraphics[width=\linewidth]{images/fig_01}
%\end{center}
%
%
%\section*{Exercice 2 -- Résistance équivalente}
%\textit{Déterminer la résistance équivalente du montage suivant.}
%\begin{center}
%\includegraphics[width=\linewidth]{images/fig_05}
%\end{center}
%




Pour aller rechercher des produits dans leurs rayons, Amazon utilise des axes linéaires afin de déplacer un préhenseur.
\begin{center}
\includegraphics[width=\linewidth]{fig_11}
\end{center}

Les performances dynamique de l'axe demandées sont les suivantes : 
\begin{itemize}
\item vitesse linéaire maximale : $50 \; \text{m}\,\text{min}^{-1}$;
\item accélération linéaire maximale : $9,8 \; \text{m}\, \text{s}^{-2}$.
\end{itemize}

La loi de commande suivie par l'axe est un trapèze de vitesse. Dans le cas d'un système à un seul axe, l'accélération maximale est toujours atteinte, la vitesse maximale, non.

\begin{obj}
L'objectif de ce travail est de déterminer les caractéristiques du moteur (vitesse et couple) permettant d'atteindre ces performances.
\end{obj}

\question{Quelle est la vitesse maximale que l'axe peut atteindre en  $\text{m}\, \text{s}^{-1}$.}
\ifprof
\begin{corrige}
$V = 0,83 \, \text{ms}^{-1}$
\end{corrige}
\else
\fi

\question{Combien de temps l'axe met-il pour atteindre la vitesse maximale ?}
\ifprof
\begin{corrige}
$T_a =0,83/9,8 = 0,08 s$
\end{corrige}
\else
\fi

\question{Quelle distance l'axe parcourt-il pour atteindre la vitesse maximale ?}
\ifprof
\begin{corrige}
\end{corrige}
\else
\fi


\question{Quelle est la longueur minimale à commander pour que l'axe puisse atteindre la vitesse maximale ?}
\ifprof
\begin{corrige}
\end{corrige}
\else
\fi

\question{Donner les profils de position, vitesse et accélération pour réaliser \SI{5}{cm}.}
\ifprof
\begin{corrige}
\end{corrige}
\else
\fi

\question{Tracer le profil de la position, de la vitesse et de l'accélération pour parcourir une distance de 50 cm. On cherchera à atteindre les performances maximales de l'axe. }
\ifprof
\begin{corrige}
\end{corrige}
\else
\fi


Un motoréducteur permet d'entraîner un système poulie -- courroie permettant de déplacer la charge. On considère :
\begin{itemize}
\item une charge de masse $1\; \text{kg}$;
\item un poulie de rayon $5\; \text{cm}$;
\item un réducteur de rapport de transmission $1:20$.
\end{itemize}

\begin{center}
\includegraphics[width=.9\linewidth]{fig_12}
\end{center}

\question{Déterminer le couple à fournir par la poulie pour déplacer la charge lorsque l'accélération est au maximum. }
\ifprof
\begin{corrige}
\end{corrige}
\else
\fi

\question{Déterminer la vitesse et le couple à fournir par le moteur en considérant que l'inertie du motoréducteur est négligeable. }
\ifprof
\begin{corrige}
\end{corrige}
\else
\fi

\question{Donner la méthode permettant de prendre en compte l'inertie $J$ du motoréducteur ? Quel serait l'impact de la prise en compte de cette hypothèse ? }
\ifprof
\begin{corrige}
\end{corrige}
\else
\fi


\end{multicols}