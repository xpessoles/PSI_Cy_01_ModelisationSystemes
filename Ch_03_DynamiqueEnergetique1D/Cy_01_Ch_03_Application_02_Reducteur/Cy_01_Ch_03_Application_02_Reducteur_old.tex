%%%% Paramétrage du TD %%%%
\def\xxactivite{ \ifprof {Application 2 -- Corrigé } \else  Application 2 \fi} % \normalsize \vspace{-.4cm}
\def\xxauteur{\textsl{Xavier Pessoles}}


%\def\xxnumchapitre{Chapitre 1 \vspace{.2cm}}
%\def\xxchapitre{\hspace{.12cm} Introduction à la dynamique du solide indéformable}

\def\xxcompetences{%
\textsl{%
\textbf{Savoirs et compétences :}
\begin{itemize}[label=\ding{112},font=\color{bleuxp}] 
\item \textit{C1-05} : Proposer une démarche permettant la détermination d’une action mécanique inconnue ou d'une loi de mouvement.
\item \textit{C2-08} : Déterminer les actions mécaniques en dynamique dans le cas où le mouvement est imposé.
\item \textit{C2-09} : Déterminer la loi de mouvement dans le cas où les efforts extérieurs sont connus.
\end{itemize}
}}
\def\xxtitreexo{Application -- Détermination de l'inertie équivalente de réducteurs}
\def\xxsourceexo{\hspace{.2cm}}%\footnotesize{Concours Commun Mines Ponts 2016}}

%\def\xxauteur{\textsl{Xavier Pessoles}}

\def\xxfigures{
%\includegraphics[width=.6\linewidth]{fig_00}
}%figues de la page de garde


\iflivret
\input{\repRel/Style/pagegarde_TD}
\else
\input{../../style/new_pagegarde}
\fi
\setlength{\columnseprule}{.1pt}

\pagestyle{fancy}
\thispagestyle{plain}

\ifprof
\vspace{5.5cm}
\else
\vspace{5.5cm}
\fi

\def\columnseprulecolor{\color{bleuxp}}
\setlength{\columnseprule}{0.4pt} 
\setcounter{numques}{0}
%%%%%%%%%%%%%%%%%%%%%%%



\ifprof
\else
\begin{multicols}{2}
\fi


\section*{Exercice 1 -- Calcul de l'inertie équivalente d'un train simple}
\setcounter{subparagraph}{0}
On donne un train d'engrenages simple avec $Z_1$, $Z_{21}$, $Z_{23}$ et $Z_3$ le nombre de dents des roues dentées. On nomme $k_1$ le rapport du train de $S_1$ et $S_2$ avec $k_1=\dfrac{\omega(2/0)}{\omega(1/0)}$ et  
$k_2$ le rapport de $S_2$ et $S_3$ avec $k_2=\dfrac{\omega(3/0)}{\omega(2/0)}$. 

On applique en entrée, sur l'arbre 1, un couple moteur $C_m\vect{z_0}$ destiné à entraîner une charge, sur l'arbre 3, modélisée par un couple résistant  $C_r\vect{z_0}$

\ifprof
\begin{center}
\includegraphics[width=.5\linewidth]{red_01}
\end{center}
\else
\begin{center}
\includegraphics[width=\linewidth]{red_01}
\end{center}
\fi
On rappelle que pour les engrenages à denture droite $d=mz$ avec $d$ le diamètre primitif, $m$ le module, $z$ le nombre de dents du pignon. $\omega(1/0)$, $\omega(2/0)$ et $\omega(3/0)$ sont les vitesses de rotation de $S_1$, $S_2$ et $S_3$ autour des axes $\left(O_1,\vect{x_g}\right)$, $\left(O_2,\vect{x_g}\right)$ et $\left(O_3,\vect{x_g}\right)$. Le repère galiléen $\mathcal{R}_g$ est lié au solide $S_0$. Les liaisons pivots sont supposées parfaites. Les moments d'inertie sont définies aux centres de masse $G_1=O_1$, $G_2=O_2$ et $G_3=O_3$ associées aux solides $S_1$, $S_2$ et $S_3$ suivant l'axe $\vect{z_0}$ sont de notés $J_1$, $J_2$ et $J_3$. 

Le train d'engrenage est entrainé par un couple moteur $C_m$ agissant sur la liaison pivot entre 1 et 0. 
Une poulie de rayon $R$ est placée sur l'extrémité droite de l'arbre 3. Une charge de masse $M$ y est suspendue. 

\question{Déterminer le rapport de réduction du train d'engrenages.}

\question{Déterminer l'inertie équivalente du réducteur seul ramené à l'axe moteur.}

\question{Déterminer l'inertie équivalente de l'ensemble réducteur et charge ramené à l'arbre moteur.}

\question{Déterminer la relation entre le couple d'entrée et le couple de sortie du réducteur.}

\question{Déterminer la relation entre le couple d'entrée, les grandeurs inertielles et l'accélération de l'arbre 1.}

\ifprof
\else
\end{multicols}
\fi
