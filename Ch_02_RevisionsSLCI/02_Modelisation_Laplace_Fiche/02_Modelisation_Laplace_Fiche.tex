%\setchapterimage{Fond_SLCI.png}
%\setchapterpreamble[u]{\margintoc}
%\chapter{Modéliser les systèmes asservis -- Transformée de Laplace}

\section{Modéliser les systèmes asservis -- Transformée de Laplace}[Transformée de Laplace]
\subsection{Définitions}

\begin{defi} [Conditions de Heavisde -- Fonction causale -- Conditions initiales nulles] 

Une fonction temporelle $f(t)$ vérifie les conditions de Heaviside lorsque les dérivées successives nécessaires à la résolution de l'équation différentielle sont nulles pour $t={0^{+}}$ :
$$
f({0^{+}})=0 \quad \dfrac{\dd f({0^{+}})}{\dd t} = 0 \quad \dfrac{\dd^2f({0^{+}})}{\dd t^2} = 0 ...
$$
On parle de conditions initiales nulles.
\end{defi}

\begin{defi} [Transformée de Laplace]
À toute fonction du temps $f(t)$, nulle pour $t\leq0$ (fonction causale), on fait correspondre une fonction $F(p)$ de la variable complexe $p$ telle que :
$$
\mathcal{L}\left[f(t)\right] = F(p)=\int\limits_{0^{+}}^\infty f(t)e^{-pt}\text{d}t.
$$
On note $\mathcal{L}\left[f(t)\right]$ la transformée directe et $\mathcal{L}^{-1}\left[F(p)\right]$ la transformée inverse.

De manière générale on note 
$\mathcal{L}\left[f(t)\right] = F(p)$,
$\mathcal{L}\left[e(t)\right] = E(p)$,
$\mathcal{L}\left[s(t)\right] = S(p)$,
$\mathcal{L}\left[\omega(t)\right] = \Omega(p)$,
$\mathcal{L}\left[\theta(t)\right] = \Theta(p)$ ...
\end{defi}



\marginnote{En dehors des conditions de Heaviside, la transformée de Laplace d'une dérivée première est donnée par $\mathcal{L}\left[ \dfrac{\dd f(t)}{\dd t}\right] =pF(p)-f(0^+)$.}

\begin{resultat} [Dérivation]
%La transformée de Laplace d'une dérivée seconde est donnée par : 
%$\mathcal{L}\left[ \dfrac{\d^2f(t)}{\d t^2}\right] =p^2F(p)-pf(0^+)-\dfrac{\d f(0^+)}{\d t}$

Dans les conditions de Heaviside :
$\mathcal{L}\left[ \dfrac{\text{\textbf{d}} f(t)}{\text{\textbf{d}} t}\right] =pF(p)$,
$\mathcal{L}\left[ \dfrac{\text{\textbf{d}}^2f(t)}{\text{\textbf{d}} t^2}\right] =p^2F(p) $,
$\mathcal{L}\left[ \dfrac{\text{\textbf{d}}^nf(t)}{\text{\textbf{d}} t^n}\right] =p^nF(p).$

\end{resultat}



\begin{table*}[!h]
\centering
\begin{tabular}{|c|c||c|c|}
\hline
Domaine temporel $f(t)$ & Domaine de Laplace $F(p)$ & 
Domaine temporel $f(t)$ & Domaine de Laplace $F(p)$ \\
\hline
\hline
Dirac $\delta(t)$ &
$\mathbf{F(p)=1}$ &
Échelon $ u(t)=k $&
$ \mathbf{U(p) = \dfrac{k}{p}}$
\\
\hline
%Créneau $\forall t\in ]0,t_1 [ \quad f(t)= A$ & 
Fonction linéaire $f(t)=t$& 
$\mathbf{F(p) =\dfrac{1}{p^2} }$ &
Puissances
$f(t) = t^n\cdot u(t)$ &
$F(p)=\dfrac{n!}{p^{n+1}} $
\\
\hline
$f(t) = \sin \left( \omega_0 t\right) \cdot u(t)$ &
$F(p) = \dfrac{\omega_0}{p^2+\omega_0^2} $ &
$f(t) = \cos \left( \omega_0 t\right) \cdot u(t)$ & 
$F(p) = \dfrac{p}{p^2+\omega_0^2} $ \\
\hline
$f(t)= e^{-at}\cdot u(t)$ & 
$F(p)= \dfrac{1}{p+a}$ &
$f(t) = e^{-at}\sin\left( \omega_0 t\right) \cdot u(t)$ &
$F(p)=\dfrac{\omega_0}{\left( p+a\right)^2 + \omega_0^2}$  \\
\hline
%$f(t)$ est $T$ périodique &
%$F(p)= \dfrac{\mathcal{L} \left[f_0 (t)\right]}{1-e^{-Tp}} \cdot u(t)$ &
$f(t)=t^ne^{-at}u(t)$ & $F(p)=\dfrac{n!}{\left( p+a\right)^{n+1}}$
& &
\\
\hline
\end{tabular}
\end{table*}


\subsection{Théorèmes}

\begin{multicols}{2}
\begin{theoreme}[Valeur initiale]
$$ \lim\limits_{t \to 0^+} f(t) = \lim\limits_{p\in\mathbb{R}, p \to \infty} pF(p)$$
\end{theoreme} 

\begin{theoreme}[Valeur finale]
$$\lim\limits_{t \to \infty} f(t) = \lim\limits_{p\in\mathbb{R}, p \to 0} pF(p)$$
\end{theoreme} 

\begin{theoreme}[Retard]
$$\mathcal{L}\left[ f\left(t-t_0\right) \right] = e^{-t_0 p}F(p)$$
\end{theoreme} 

\begin{theoreme}[Amortissement]
$$\mathcal{L} \left[ e^{-a t} f\left(t\right) \right] = F(p+a)$$
\end{theoreme} 
\end{multicols}





