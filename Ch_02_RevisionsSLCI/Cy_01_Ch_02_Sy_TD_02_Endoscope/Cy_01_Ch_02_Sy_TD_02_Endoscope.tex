\setchapterimage{fig_00.jpg}
\chapter*{TD \arabic{cptTD} \\ 
Robot pour la chirurgie endoscopique -- \ifprof Corrigé \else Sujet \fi}

\addcontentsline{toc}{section}{TD \arabic{cptTD} : Robot pour la chirurgie endoscopique -- \ifprof Corrigé \else Sujet \fi}

\iflivret \stepcounter{cptTD} \else
\ifprof  \stepcounter{cptTD} \else \fi
\fi
\setcounter{question}{0}

\marginnote{Banque PT -- SIA -- 2005.}
%\marginnote{\UPSTIcompetence[2]{B2-07}\UPSTIcompetence[2]{C2-03}}
\begin{marginfigure}
\centering
\includegraphics[width=\linewidth]{fig_00}
\end{marginfigure}

\section*{Présentation}
\ifprof
\else
On s’intéresse au robot de chirurgie endoscopique <<~Endoxirob~>>. Il est nécessaire de permettre à l’instrument chirurgical de se mouvoir avec des performances dynamiques comparables ou meilleures que celles réalisées par un chirurgien humain. 
L’étude est effectuée uniquement pour l’axe « d’élévation » selon $\vect{z_1}$. 


\begin{table*}[!h]
\centering
\begin{tabular}{cp{10cm}p{3cm}}
\hline 
Req. & Exigence & Niveaux \\ \hline
1 & Vitesse de translation nominale & $>\SI{0,1}{m.s^{-1}}$ \\ 
2 & Temps $t_1$ pour atteindre la vitesse nominale & $\SI{100}{ms}$ maximum \\ 
3 & Précision -- Écart statique & $<\SI{0,2}{mm}$ \\ 
4 & Précision --  Retard de traînage pour une rampe de $ \SI{0,1}{m.s^{-1}}$  & $<\SI{0,1}{s}$ \\ 
5 & Stabilité -- Dépassement & Aucun \\ 
6 & Stabilité -- Marge de phase & $45\degres$ \\ 
7 & Rapidité -- Bande passante à \SI{-3}{dB} pour la partie mécanique & $\SI{4}{Hz}$ \\
\hline
\caption{Liste partielle des exigences.}
\end{tabular}
\end{table*}
La figure suivante présente le schéma-blocs de l'axe d’élévation (selon $\vect{Z_1}$) du dispositif de commande de l’instrument chirurgical. 

\begin{center}
\includegraphics[width=\linewidth]{fig_01}
\end{center}

La console permet de capter le déplacement de la main, de le coder, de le corriger éventuellement afin d’élaborer la consigne de position angulaire du rotor moteur. 
La position angulaire est ensuite transformée en position linéaire de l’instrument par un mécanisme de transformation de mouvement à crémaillère.


\begin{marginfigure}
\includegraphics[width=\linewidth]{fig_02}
\end{marginfigure}

La figure ci-contre présente de façon simplifiée, la chaîne cinématique de l’axe d’élévation :
\begin{itemize}
\item l’actionneur est un moto-réducteur (1) à courant continu Gammatic n° RH-8D-6006;
le premier étage de transmission du mouvement se fait par une courroie crantée (2) qui s’enroule sur des poulies de même diamètre pour entraîner l’arbre intermédiaire (3);
\item une roue dentée (3’) de diamètre primitif $\Phi_1 = \SI{38,4}{mm}$ solidaire de l’arbre intermédiaire (3) engrène avec une crémaillère (4) solidaire de la partie supérieure mobile du robot;
\item cette crémaillère est en liaison glissière d’axe $\vect{z_1}$ par rapport à la partie inférieure du robot considérée comme fixe dans cette partie du sujet, et notée « Bâti ».
\end{itemize}

Pour équilibrer le poids de la partie supérieure (coulisseau compris) de masse $M = \SI{5,5}{kg}$, les ingénieurs ont placé un contre-poids (7) de masse $2M = \SI{11}{kg}$ tiré par un câble qui s’enroule sur un tambour (3") de diamètre $\Phi_2 = \dfrac{\Phi}{2} = \SI{19,2}{mm}$ solidaire de  l’arbre intermédiaire (3).




Le constructeur du moto-réducteur donne les caractéristiques mécaniques en sortie du réducteur  ce qui permet de considérer le moto-réducteur comme un « moteur qui tourne lentement avec un couple élevé » :
\begin{itemize}
\item puissance nominale en sortie du réducteur : $P_{\text{nom}} = \SI{8,6}{W}$;
\item couple nominal en sortie du réducteur : $C_{\text{nom }} = \SI{1,4}{Nm}$;  
\item couple de maintien en sortie du réducteur : $C_{\text{maint}} = \SI{1,5}{Nm}$;
\item couple maxi en sortie du réducteur : $C_{\text{max}}	= \SI{2,7}{Nm}$;
\item vitesse nominale en sortie du réducteur : $N_{\text{nom}} = \SI{60}{tr.min^{-1}}$;
\item vitesse maxi en sortie du réducteur : $N_{\text{max}} = \SI{100}{tr.min^{-1}}$;
\item moment d’inertie total ramené à l’arbre de sortie du réducteur : $J_1 = \SI{3,70e-3}{kg.m^2}$; 
\item capteur de position : codeur 360 incréments par tour monté sur le rotor du moteur;
\item rapport de réduction : 50 (n’interviendra que dans le calcul de la résolution du capteur).
\end{itemize}
\fi

\subsection*{Évaluation de la fonction de transfert du moto-réducteur}
\ifprof
\else
Le constructeur donne les caractéristiques électro-mécaniques exprimées à la sortie du réducteur.
On rappelle les équations temporelles :
$u(t) = R i(t) + L \dfrac{\dd i(t)}{\dd t} + e(t)$, $e(t) = k_e  \omega_{\text{réd}}(t)$, 
$C_{\text{réd}} (t)= k_c  i(t)$, $C_{\text{réd}} (t) - C_r - f_v \omega_{\text{réd}}(t)= J_{\text{équ}} \dfrac{\dd \omega_{\text{réd}}(t)}{\dd t}$. 

\marginnote[-3cm]{Avec :
\begin{itemize}
\item $u(t)$ tension appliquée aux bornes de l’induit 
\item $i(t)$ intensité du courant traversant l’induit
\item $e(t)$ force électromotrice induite par la rotation du moteur, évaluée à la  sortie du réducteur;
\item $\omega_{\text{réd}}(t)$ vitesse de rotation à la sortie du réducteur;
\item $C_{\text{réd}} (t)$ couple moteur ramené à la sortie du réducteur;
\item $R = \SI{10}{\Omega}$ : résistance de l’induit;
\item $L = \SI{2,2}{mH}$ inductance de l’induit;
\item $k_c = \SI{2,1}{N.m.A^{-1}}$ constante de couple évaluée à la sortie du réducteur;
\item $k_e = \SI{2,1}{V.s.rad^{-1}}$ constante de f.e.m évaluée à la sortie du réducteur;
\item $C_r = \SI{0,2}{N.m}$ couple résistant induit par les frottements secs, ramené à la sortie du réducteur 
\item $f_v = \SI{0,04}{Nm.s.rad^{-1}}$ coefficient de frottement visqueux équivalent à toutes les pièces en mouvement, ramené à la sortie du réducteur 
\item $J_{\text{équ}} = \SI{7e-3}{kg.m^2}$ moment d’inertie équivalent à toutes les pièces en mouvement, ramené à la sortie du réducteur. 
\end{itemize}}

Les transformées de Laplace des fonctions $u(t)$, $i(t)$, $e(t)$, $\omega_{\text{réd}}(t)$,  $C_{\text{réd}}(t)$ sont respectivement $U(p)$, $I(p)$, $E(p)$, $\Omega_{\text{réd}}(p)$ et $C_{\text{réd}}(p)$. On considère toutes les conditions initiales nulles.

\fi

\question{Transformer les équations temporelles ci-dessus. 
Remplir sous forme littérale les blocs du schéma suivant. Exprimer les grandeurs physiques entre chaque bloc. }
\ifprof
\begin{corrige}
On a : 
\begin{itemize}
\item $U(p) = R I(p) + LpI(p)+ E(p)$;
\item $E(p) = k_e  \Omega_{\text{réd}}(p)$;
\item $C_{\text{réd}} (p)= k_c  I(p)$;
\item $C_{\text{réd}} (p) - C_r(p) - f_v \Omega_{\text{réd}}(p)= J_{\text{équ}}  p \Omega_{\text{réd}}(p)$. 
\end{itemize}
\end{corrige}
\begin{center}
\includegraphics[width=\linewidth]{cor_01}
\end{center}

\else
\fi
\ifprof
\else
\begin{center}
\includegraphics[width=\linewidth]{fig_03}
\end{center}
\fi

\question{Exprimer littéralement sous forme canonique la fonction de transfert du moto-réducteur $M(p)=\dfrac{\Omega_{\text{réd}}(p)}{U(p)}$ lorsque $C_r(p)=0$.}
\ifprof
\begin{corrige}
En raisonnant à partir des équations, on a : 
$U(p) = R I(p) + LpI(p)+ E(p)$ 
$\Rightarrow U(p) = I(p) \left(R + Lp\right)+ k_e  \Omega_{\text{réd}}(p)$
$\Rightarrow U(p) = \dfrac{C_{\text{réd}}}{k_c} \left(R + Lp\right)+ k_e  \Omega_{\text{réd}}(p)$
$\Rightarrow U(p) = \dfrac{\left( J_{\text{équ}}  p +f_v\right)\Omega_{\text{réd}}(p)}{k_c} \left(R + Lp\right)+ k_e  \Omega_{\text{réd}}(p)$

$\Rightarrow U(p) = \left(\dfrac{\left( J_{\text{équ}}  p +f_v\right)}{k_c} \left(R + Lp\right)+ k_e \right) \Omega_{\text{réd}}(p)$

Au final, $M(p)=\dfrac{k_c}{ \left( J_{\text{équ}}  p +f_v\right) \left(R + Lp\right)+ k_e k_c}$.

En mettant l'expression sous forme canonique, on a :
$M(p)=\dfrac{k_c}{   J_{\text{équ}}   Lp^2 +\left( RJ_{\text{équ}}    +  L f_v\right) p + k_e k_c+Rf_v}$

$\Rightarrow  M(p)=\dfrac{\dfrac{k_c}{k_e k_c+Rf_v}}{  \dfrac{ J_{\text{équ}}   L}{k_e k_c+Rf_v}p^2 +\dfrac{ RJ_{\text{équ}}    +  L f_v}{k_e k_c+Rf_v}p +1 }$.

En réalisant l'application numérique, on a : 
$M(p)=\dfrac{0,44}{  3,2\times 10^{-6} p^2 +14,6\times 10^{-3}p +1 }$.

\end{corrige}
\else
\fi

\ifprof
\else
Quel que soit le résultat obtenu à la question précédente, on utilisera l’expression de $M(p)$ suivante : $M_1(p)=\dfrac{0,436}{1+14,5\cdot 10^{-3}p+3,1\cdot 10^{-6}p^2}$. 

Sur le système de levage non asservi c’est à dire avec le capteur de position angulaire déconnecté, on a pratiqué, un essai en charge, en donnant au moteur un échelon de tension $u(t) = \SI{24}{V}$. 
Avec une génératrice tachymétrique dont le gain est de $\SI{0,166}{V.s.rad^{-1}}$ , on a tracé la courbe de tension image de $\omega_{\text{réd}}(t)$.  

\begin{marginfigure}
\includegraphics[width=\linewidth]{fig_04}
\end{marginfigure}


Remarque : la partie supérieure du robot est supposée rigide pendant cet essai.
\fi

\question{Après avoir analysé cette courbe, expliquer pourquoi on peut négliger l’inductance $L$.}
\ifprof
\begin{corrige}
En observant cette courbe, l'absence de tangente horizontale à l'origine permet de modéliser le système comme un système d'ordre 1.  En négligeant $L$, le terme d'ordre 2 devient donc négligeable et $M(p)$ se modélise par une fonction d'ordre 1. 
\end{corrige}
\else
\fi


\question{Justifier analytiquement la réponse précédente à partir de l’expression de $M_1(p)$ lorsque l’on envisage une étude fréquentielle : on précisera la valeur du pôle dominant, l’autre (faisant intervenir la valeur de $L$) étant rejeté.}
\ifprof
\else
\begin{remarque}
Un pole est dit dominant par rapport à un autre quand sa partie réelle est grande devant l'autre.
\end{remarque}
\fi

\ifprof
\begin{corrige}
En utilisant la formulation de $M_1(p)$ donnée, on calcule le discriminant du dénominateur et on a : 
$\Delta = \left(14,5\times 10^{-3}\right)^2 - 4\cdot 3,1 \times 10^{-6} = 0,00019785 $. 
Au final, $p_1 = \dfrac{-14,5\times 10^{-3}-\sqrt{\Delta}}{2\cdot 3,1 \times 10^{-6}}\simeq -4607$ et $p_2 = \dfrac{-14,5\times 10^{-3}+\sqrt{\Delta}}{2\cdot 3,1 \times 10^{-6}}\simeq -70$. 
Le dénominateur peut donc se factoriser sous la forme $ 3,1 \times 10^{-6}\left( p + 4607\right)\left(p+70\right)$.

Le pole $p_2$ et donc dominant par rapport à $p_1$.
\end{corrige}
\else
\fi

\ifprof
\else
\textbf{Dans la suite du problème, on néglige l’inductance $L$ du moteur.}
\fi

\question{Exprimer littéralement, sous forme canonique, la fonction de transfert du moto-réducteur  $M_2(p)=\dfrac{\Omega_{\text{réd}}(p)}{U(p)}=\dfrac{G_s}{1+Tp}$.  
Donner les valeurs numériques de $G_s$ et de $T$ à partir de l’expression de $M_1(p)$ et des réponses apportées précédemment.}
\ifprof
\begin{corrige}
En utilisant l'expression établie initialement en en négligeant l'inductance, on a 
$  M_2(p)=\dfrac{\dfrac{k_c}{k_e k_c+Rf_v}}{  \dfrac{ RJ_{\text{équ}}  }{k_e k_c+Rf_v}p +1 }$.
En réalisant l'application numérique, $M_2(p)=\dfrac{0,437}{1+0,015p}$ 
($G_s = \SI{0,437}{rad.s^{-1}.V^{-1}}$ 
et $T=\SI{0,015}{s}$).
\end{corrige}
\else
\fi


\question{Déterminer les valeurs de $G_s$ et $T$, à partir de la courbe de tension image de $\omega_{\text{réd}}(t)$  (expliquer les démarches sous la figure et comparer avec les résultats obtenus précédemment). }
\ifprof

\begin{marginfigure}
\includegraphics[width=\linewidth]{cor_02}
\end{marginfigure}

\begin{corrige}
La tension de consigne étant de \SI{24}{V} et la vitesse de sortie est telle que  $24\cdot G_S \cdot K_{\text{tachy}}= \SI{1,75}{V}$ soit $G_S = \dfrac{1,75}{24\times 0,166} = \SI{0,439}{rad.s^{-1}.V^{-1}} $. En utilisant la méthode de 63\,\% de la valeur finale, on a $\tau = \SI{0,02}{s}$.
On constate que les résultats sont relativement proches de ceux formulés par l'hypothèse <<~$L$ négligeable~>>.
\end{corrige}

\else
\fi

\subsection*{Respect du critère de marge de phase}
\ifprof
\else
La boucle d’asservissement de la position angulaire de l’arbre de sortie du réducteur est définie par le schéma-blocs figure suivante. 
La consigne de position en incréments est élaborée par le  calculateur, à partir des informations envoyées par la console.

\begin{center}
\includegraphics[width=\linewidth]{fig_05}
\end{center}


Le convertisseur-amplificateur $K$ de gain $k$ variable élabore la commande du moteur.
Le codeur incrémental $C$ placé sur le rotor du moteur a une résolution de 360 incréments par tour. Il est associé à un compteur -- décompteur qui élabore la mesure de position en incréments. 
Le système est discret (non continu) mais on l’assimile à un système continu car le comptage est très rapide. 
Le réducteur a un rapport de réduction de 50.
 
 \fi
 
\question{Donner la fonction de transfert du bloc $B(p)$ et la valeur du coefficient du bloc $C$ en incr./rad.  
Exprimer numériquement, en fonction de $k$, la fonction de transfert en boucle ouverte $H_O(p)$ .}
\ifprof
\begin{corrige}
$B$ assure la réduction de la fréquence de rotation et son intégration dans le but d'obtenir un angle; donc $B(p)=\dfrac{1}{50p}$.
Par ailleurs $C=\dfrac{360}{2\pi}\text{ incr/rad} $. 

On a donc $H_O(p)=\dfrac{1}{50p}\dfrac{360}{2\pi} \dfrac{kG_S}{1+Tp}$.
\end{corrige}
\else
\fi


\question{Tracer les diagrammes de Bode du système en boucle ouverte pour $k = 1$.
Le système est-il stable en boucle fermée pour cette valeur de $k$ ? Justifier.
} 
\ifprof
\else
\begin{remarque}
Au vu de l'exigence 6, on admet que le système est stable en boucle fermé si, sur le tracé de Bode de la boucle ouverte, lorsque le gain est nul, la phase est supérieure à $-180+45=-135\degres$. 
\end{remarque}
\fi

\ifprof
\begin{center}
\includegraphics[width=.7\linewidth]{Q8.png}
\end{center}
\begin{corrige}
Au vu du tracé, la phase est supérieure à $-135\degres$ lorsque le gain est nul. Le système est donc stable.
\end{corrige}
\else
\fi

%********
%On a tracé sur le document réponses figure R28 le diagramme de Black du système en boucle ouverte pour k = 1 lorsque la valeur de $L$ n’est pas négligée.
% 
%Question 28. : 
%Quelle est l’influence de la prise en compte de $L$ sur la stabilité en boucle fermée ? (répondre sur le document réponses sous la figure R28). L’hypothèse consistant à négliger $L$ est-elle vérifiée ?
%Déterminer, et expliquer à partir de constructions faites sur le diagramme de Black document réponses figure R28, la valeur $k_{45}$ de $k$ qui permet d’obtenir la marge de phase de 45\degres  spécifiée dans le cahier des charges.
%*********

%On admet que le système est stable pour $k_{45}=0,0767$.

\question{Calculer %, pour la valeur $k_{45}$ de $k$ établie précédemment, 
l’écart statique $\varepsilon_{\text{cons} \infty}$ en incréments lorsque la consigne est un échelon de position : $\text{Cons}(t) = 1\cdot u(t)$.}
\ifprof
\begin{corrige}~\\
\textbf{Méthode 1 (à connaître après le cours sur la précision -- Cycle 2)}

La boucle ouverte est de classe 1, l'entrée est un échelon et il n'y a pas de perturbation. L'écart statique est donc nul.


\textbf{Méthode 2 (à savoir faire) -- Calcul de l'écart}

On a $\varepsilon(p)=\dfrac{\text{Cons}(p)}{1+\text{FTBO}(p)}$. $\text{Cons}(p)=\dfrac{1}{p}$.
On a alors 
$\varepsilon_{\text{cons} \infty} $
$= \lim\limits_{t\to+\infty} \varepsilon(t)$ 
$ = \lim\limits_{p\to+0} p\varepsilon(p)$ 
$ = \lim\limits_{p\to+0} p\dfrac{1}{p}\dfrac{1}{1+\text{FTBO}(p)}=0$.
 
% \textbf{Méthode 3}



\end{corrige}
\else
\fi

\ifprof
\else
Étant donné que la perturbation $C_r$ intervient entre deux blocs de $M(p)$, on adopte le schéma de la 
figure suivante pour faciliter les calculs de la question suivante. 

\begin{center}
\includegraphics[width=\linewidth]{fig_06}
\end{center}

\fi


\ifprof
\newpage
\else
\fi

\question{Calculer, pour la valeur $k_{45}$ de $k$ établie précédemment, l’écart statique $\varepsilon_{\text{pert} \infty}$ en incréments entre la consigne et la mesure lorsque la perturbation est l’échelon de couple résistant $C_r u(t)$ induit par les frottements secs. }
\ifprof
\begin{corrige}
On  a $\varepsilon(p)=\text{Cons}(p)-B(p)C\Omega(p)$ $=\text{Cons}(p)-B(p)C\left( \varepsilon(p) K M(p)- C_r(p) \dfrac{M(p)R}{k_c} \right)$ $ \Leftrightarrow \varepsilon(p) \left(1+ B(p)C  K M(p)\right)=\text{Cons}(p) +B(p)C C_r(p) \dfrac{M(p)R}{k_c}  $
$ \Leftrightarrow \varepsilon(p) =\text{Cons}(p)\dfrac{1}{1+ B(p)C  K M(p)}  $

$+ C_r(p) \dfrac{M(p)R B(p)C}{k_c\left( 1+ B(p)C  K M(p) \right)}  $.

$=\lim\limits_{p\to 0} p \varepsilon (p) = $
$ =\lim\limits_{p\to 0}  \dfrac{1}{1+ B(p)C  K M(p)}  $

$+ C_r \dfrac{M(p)R B(p)C}{k_c\left( 1+ B(p)C  K M(p) \right)}  $

$ =\lim\limits_{p\to 0}  C_r \dfrac{\dfrac{G_S}{1+Tp}R \dfrac{1}{50p}C}{k_c\left( 1+ \dfrac{1}{50p}C  K \dfrac{G_S}{1+Tp} \right)}  $


$ = \lim\limits_{p\to 0}  C_r \dfrac{G_S R C}{k_c\left( \left(1+Tp \right)50p+ C  K G_S \right)}  $

$= C_r \dfrac{G_S R C}{k_c  C  K G_S }= C_r \dfrac{ R }{k_c   K }  $.

On a donc  $\varepsilon_{\text{pert} \infty} = C_r \dfrac{ R }{k_c   K }  $ soient $\varepsilon_{\text{pert} \infty} = 0,2 \dfrac{ 10}{2,1   1 } =\SI{0,95}{incr}$.


\end{corrige}
\else
\fi




\question{La chaîne cinématique de transmission est telle qu’il faut 150 incréments pour que la crémaillère se déplace de $\SI{1}{mm}$, quelle est l’incidence de cet écart sur la position de l’instrument ? Conclure par rapport aux exigences du cahier des charges.
Proposer une modification du bloc $K$ qui annulerait cet écart.}
\ifprof
\begin{corrige}
Si on majore l'erreur précédente à 1 incrément, l'erreur sur la position de l’instrument est de \SI{0,007}{mm}.
Cette erreur est inférieure à \SI{0,2}{mm} (exigence 3).  On peut conserver la valeur $k=1$.
\end{corrige}
\else
\fi



\subsection*{Vérification des performances de la chaîne de positionnement de l'instrument}
\subsubsection*{Modélisation par schéma-blocs}
\ifprof
\else
\begin{table*}[!h]
\centering
\includegraphics[width=\linewidth]{fig_07}
\end{table*}
\fi

\subsubsection*{Analyse du déplacement en translation de la crémaillère}
\ifprof
\else
Lorsque la boucle d’asservissement est bien réglée, la fonction de transfert est : $H_1(p)=\dfrac{\Theta(p)}{\text{Cons}(p)} = \dfrac{0,00035}{1+0,014p+0,00017 p^2}$. 
On rappelle que la courroie s’enroule sur des poulies de même diamètre et que la roue dentée qui engrène avec la crémaillère a un diamètre $\Phi_1 = \SI{38,4}{mm}$.
\fi

\question{Exprimer le coefficient du bloc $H_2$ et préciser l’unité.}
\ifprof
\begin{corrige}
En notant $x$ le déplacement en translation, on a $x=\dfrac{38,4}{2} \Theta$. On a donc $H_2 =\dfrac{x}{\Theta}=\SI{19,2}{mm.rad^{-1}}=\SI{0,0192}{m.rad^{-1}}$.   
\end{corrige}
\else
\fi

En régime statique, la position de la crémaillère est l’image de la position de la main, aux écarts près. 

\question{Quelle relation doit vérifier le produit $P$ des gains des blocs $C_1$ , $H_1$ , $H_2$  ? Justifier.
Exprimer le coefficient $c_1$ en incréments par mètre du bloc $C_1$ .
}
\ifprof
\begin{corrige}
Dans le cas ou l'instrument doit réaliser les mêmes mouvements que la main, il est nécessaire que $P=1$. 

On  a $G_{H_1}\cdot G_{H_2} = 0,0192 \times 0,00035 = 6,72\times 10^{-6}$.
 
On a donc $c_1 = \SI{148 810}{incr.m^{-1}}$.
\end{corrige}
\else
\fi

Pour augmenter la précision de l’opération chirurgicale, on désire que la crémaillère se déplace 10 fois moins que la main. 




\question{Exprimer le nouveau coefficient $c_2$ du bloc $C_1$ ainsi que le nouveau produit  $P_1$ . }
\ifprof
\begin{corrige}
On souhaite maintenant que $P_1=\dfrac{1}{10}$ et $c_1 = \SI{148 81}{incr.m^{-1}}$.
\end{corrige}
\else
\fi

\subsubsection*{Analyse du déplacement de l’instrument chirurgical par rapport à la crémaillère}
\ifprof
\else

\begin{marginfigure}
\includegraphics[width=\linewidth]{fig_09}
\end{marginfigure}

La partie supérieure du robot est constituée par assemblage de tubes minces en fibres de carbone. 
On modélise cette partie par deux solides : $S_0$ représentant la crémaillère et les solides qui y sont liés ;  et $S_1$ représentant l’instrument chirurgical. 
Ces solides sont considérés en liaison glissière parfaite et reliés par un ressort de raideur $k_0$ et un amortisseur de coefficient $f_0$, montés en parallèle comme le montre le schéma.

\begin{marginfigure}
\includegraphics[width=\linewidth]{fig_08}
\end{marginfigure}

Pour identifier la fonction de transfert $H_3(p)=\dfrac{D_{\text{instrum}}(p)}{D_{\text{crem}}(p)}$ de cette partie, on a imposé à la crémaillère un échelon de déplacement $d_{\text{crém}}(t) = \SI{20e-3}{m}$ à partir de la position d’équilibre. On a tracé la courbe de déplacement $d_{\text{instrum}}(t)$ de l’instrument.


\begin{figure}[!h]
\includegraphics[width=\linewidth]{fig_10}
\end{figure}

On donne sur la figure suivante les abaques des dépassements relatifs et des temps de réponse réduits d’un système du second ordre.



\fi

\question{Établir, à partir de cette figure, l’expression de la fonction de transfert $H_3(p)$; déterminer les valeurs caractéristiques : gain statique, coefficient d’amortissement et pulsation propre.}
\ifprof
\begin{corrige}
Pour un échelon de $\SI{20}{mm}$, le déplacement est de $\SI{20}{mm}$. Le gain statique est donc de $K=1$. 

Le premier dépassement absolu est d'environ $\SI{30}{mm}$ soit un dépassement de 50\, \% par rapport à la valeur finale. En utilisant l'abaque des temps de réponse, on trouve $\xi = 0,2$. 

Le temps de réponse à 5\, \% est d'environ $\SI{0,55}{s}$. En utilisant l'abaque, on a $\omega_0 Tr = 15$. On a donc $\omega_0=\SI{27}{rad.s^{-1}}$.

Au final, $H_3(p)=\dfrac{1}{1+\dfrac{2\cdot 0,2}{27}p+\dfrac{p^2}{27^2}}$.
\end{corrige}
\else
\fi


\question{Le critère de la bande passante de \SI{4}{Hz} à  \SI{-3}{dB} est-il satisfait ?}
\ifprof
\begin{corrige}
On trace le diagramme de Bode. La bande passante à  \SI{-3}{dB} est de \SI{4}{Hz}. Une marge de sécurité serait peut être préférable.

\begin{center}
\includegraphics[width=\linewidth]{Q16_2}
\end{center}
\end{corrige}
\else
\fi

\ifprof
\else

Les questions suivantes vont permettre de déterminer l’expression analytique de $H_3(p)$, de façon à analyser l’influence du paramètre de raideur, pour respecter le critère de bande passante du cahier des charges.

Soit $m_1$ la masse du solide $S_1$. L’axe du mouvement est vertical ascendant et noté $\vect{z}$. L’origine $O$ du mouvement de $S_1$ correspond à la position pour laquelle le ressort est à sa longueur libre.
Soit $a_0$ l’allongement du ressort dans la position d’équilibre qui prend en compte l’action de la pesanteur. 
Nota : $a_0$ est négatif car on considère qu’il y a allongement du ressort.

L'équation du mouvement du ressort autour de la position d'équilibre est donné par : $m_1\dfrac{\dd^2 z(t)}{\dd t^2} + f_0 \dfrac{\dd z (t)}{\dd t}+k_0 z = 0$.
\fi

\question{Après avoir mis l’équation différentielle du mouvement sous forme canonique : $A\ddot{z}(t)+B\dot{z}(t)+z(t)=0$, exprimer le coefficient d’amortissement $\xi_3$ et la pulsation propre $\omega_{03}$ du mouvement en fonction de $k_0$, $f_0$, $m_1$ .}
\ifprof
\begin{corrige}
On a  $m_1\dfrac{\dd^2 z(t)}{\dd t^2} + f_0 \dfrac{\dd z (t)}{\dd t}+k_0 z = 0$ 
$\Leftrightarrow  \dfrac{m_1}{k_0}\dfrac{\dd^2 z(t)}{\dd t^2} + \dfrac{f_0}{k_0} \dfrac{\dd z (t)}{\dd t}+ z = 0$. 

On a donc $\omega_{03}^2=\dfrac{k_0}{m_1}$ et $\dfrac{2\xi_3}{\omega_{03}}=\dfrac{f_0}{k_0}$ et $\xi_3 =\dfrac{f_0}{2k_0}\sqrt {\dfrac{k_0}{m_1}}$ $=\dfrac{f_0}{2}\sqrt {\dfrac{1}{k_0m_1}}$.
\end{corrige}
\else
\fi

On donne la valeur numérique de la masse de l’instrument chirurgical et de la plaque d’interface : $m_1 = \SI{1,6}{kg}$.


\question{Pour la valeur de $\omega_{03}$ calculée précédemment, déterminer la valeur minimale de la raideur $k_0$ (en N/m) qui permettrait de respecter le critère de la bande passante à $\SI{-3}{dB}$ de \SI{4}{Hz}.
(On notera que $ \omega_{-\SI{3}{dB}}> \omega_{03}$).
}
\ifprof
\begin{corrige}
On a $ \omega_{-\SI{3}{dB}}> \omega_{03}$ or $\omega_{03}^2=\dfrac{k_0}{m_1}$ $\Rightarrow k_0= m_1 \omega_{03}^2$; donc nécessairement, $ \omega_{-\SI{3}{dB}}>\sqrt{\dfrac{k_0}{m_1}}$ $\Rightarrow k_0 <\omega_{-\SI{3}{dB}} ^2 m_1 $ $\Rightarrow k_0 <2^2 4^2 \pi^2 \times 1,6 $ $k<\SI{1010}{N.m^{-1}}$.
\end{corrige}
\else
\fi

\subsubsection*{Analyse du déplacement de l’instrument par rapport au déplacement de la main}
\ifprof
\else



On conserve la valeur $c_1$ du bloc $C_1$. La fonction de transfert du système est : $H(p)=\dfrac{D_{\text{instrum}}(p)}{D_{\text{main}}(p)}=\dfrac{1}{\left(1+0,014p+0,00017p^2 \right)\left( 1+0,015p+0,0014p^2\right)}$ .
Le chirurgien impose avec sa main une rampe de déplacement  de \SI{100}{mm.s^{-1}}. On a tracé sur la figure \ref{fig:q19} les courbes de déplacement de la main et de l’instrument.

\begin{marginfigure}
\includegraphics[width=\linewidth]{Q19}
\caption{Courbes de déplacement de la main }
\label{fig:q19}
\end{marginfigure}
\fi

\question{Mettre en évidence et donner les valeurs numériques :
de l’écart dynamique maximal; de l’écart de traînage (ou de vitesse) $\varepsilon_v$ en régime établi, du retard de traînage. Le cahier des charges est-il satisfait pour ce dernier critère ?
}
\ifprof
\begin{marginfigure}
\centering
\includegraphics[width=\linewidth]{Q19_cor}
\end{marginfigure}
\begin{corrige}
\begin{itemize}
\item Écart dynamique maximal : $\SI{5}{mm}$.
\item Écart de traînage (ou de vitesse) $\varepsilon_v$ en régime établi : $\SI{4}{mm}$.
\item Retard de traînage : $\SI{0,02}{s}$ -- cahier des charges validé -- Req 4.
\end{itemize}
\end{corrige}
\else
\fi


\ifprof
\else
\begin{marginfigure}
\includegraphics[width=\linewidth]{fig_12}
\caption{Courbe de gain \label{fig:12}}
\end{marginfigure}

On donne figure \ref{fig:12}, la courbe d’amplitude (gain) de $H(p)$ pour $p = j\omega$ dans le plan de Bode. 
La main du chirurgien est prise d’un tremblement sensiblement sinusoïdal dont la période est de \SI{0,25}{s} et l’amplitude \SI{1}{mm}. 
\fi


\question{Déterminer à partir de cette courbe, l’amplitude du mouvement pris par l’instrument. Quelle est la conséquence de ce mouvement sur la plaie chirurgicale ?}
\ifprof
\begin{corrige}
Pour une sinusoïde de période $\SI{0,25}{s}$, la pulsation est de $\dfrac{2\pi}{0,25} = \SI{25}{rad.s^{-1}}$.
À cette pulsation, le gain est de \SI{8}{dB}. Le rapport $S/E$ est donc de $10^{8/20} = 2,5$ ainsi l'amplitude du robot sera de $\SI{2,5}{mm}$. Il faudrait régler l'asservissement pour que ces vibrations soient atténuées/filtrées (plutôt qu'amplifiées).
\end{corrige}
\else
\fi

\subsubsection*{Amélioration des performances dynamiques}
\ifprof
\else
On souhaite limiter l’amplitude du mouvement de l’outil lors du tremblement de la main en  filtrant le signal traité par le capteur-codeur $C_1$ de sorte que les tremblements n’apparaissent plus sur le soustracteur. On propose trois filtres du premier ordre, de gains statiques égaux à 1 et de constantes de temps :	 $T_1 = \SI{0,04}{s}$ ; $T_2 = \SI{0,1}{s}$; $T_3 = \SI{0,5}{s}$.
\fi


\question{Tracer sur la figure précédente, les trois courbes asymptotiques d’amplitude de ces filtres avec des couleurs différentes. 
Sachant que les mouvements dont la période est inférieure à \SI{1}{s} ne doivent pas être atténués de plus de \SI{1}{dB}, choisir le numéro 1, 2 ou 3 du filtre qui atténue de 8 à \SI{10}{dB} le tremblement de la main de période \SI{0,25}{s}. 
Tracer sur cette figure, dans une autre couleur, l’allure de la courbe d’amplitude corrigée par ce filtre.
Le niveau de \SI{4}{Hz}, de la bande passante à $-\SI{3}{dB}$ du critère de rapidité est-il toujours respecté ? 
}
\ifprof
\begin{center}
\includegraphics[width=\linewidth]{Q20_cor}
\end{center}
\begin{corrige}
Le filtre $T_2 = \SI{0,1}{s}$ permet d'atténuer le gain à une pulsation de $\SI{25}{rad.s^{-1}}$ sans trop atténuer le gain à des pulsation inférieures à $\SI{6,3}{rad.s^{-1}}$.
\end{corrige}
\else
\fi



\ifprof
\else

\marginnote{
\begin{solution}
\begin{enumerate}[wide, labelwidth=!, labelindent=0pt] 
 \item $U(p) = R I(p) + LpI(p)+ E(p)$; $E(p) = k_e  \Omega_{\text{réd}}(p)$; $C_{\text{réd}} (p)= k_c  I(p)$;
 $C_{\text{réd}} (p) - C_r(p) - f_v \Omega_{\text{réd}}(p)= J_{\text{équ}}  p \Omega_{\text{réd}}(p)$. 
 \item $ M(p)=\dfrac{\dfrac{k_c}{k_e k_c+Rf_v}}{  \dfrac{ J_{\text{équ}}   L}{k_e k_c+Rf_v}p^2 +\dfrac{ RJ_{\text{équ}}    +  L f_v}{k_e k_c+Rf_v}p +1 }$.
 \item .
 \item .
 \item $  M_2(p)=\dfrac{\dfrac{k_c}{k_e k_c+Rf_v}}{  \dfrac{ RJ_{\text{équ}}  }{k_e k_c+Rf_v}p +1 }$.
 \item $G_S = \SI{0,439}{rad.s^{-1}.V^{-1}} $, on a $\tau = \SI{0,02}{s}$.
 \item $B(p)=\dfrac{1}{50p}$, $C=\dfrac{360}{2\pi}\text{ incr/rad} $, $H_O(p)=\dfrac{1}{50p}\dfrac{360}{2\pi} \dfrac{kG_S}{1+Tp}$.
 \item .
 \item L'écart statique est nul.
 \item $\varepsilon_{\text{pert} \infty} = 0,2 \dfrac{ 10}{2,1   1 } =\SI{0,95}{incr}$.
 \item $H_2 =\dfrac{x}{\Theta}=\SI{19,2}{mm.rad^{-1}}=\SI{0,0192}{m.rad^{-1}}$
 \item .
 \item .
 \item .
 \item $H_3(p)=\dfrac{1}{1+\dfrac{2\cdot 0,2}{27}p+\dfrac{p^2}{27^2}}$.
 \item .
 \item $\omega_{03}^2=\dfrac{k_0}{m_1}$ et $\xi_3 =\dfrac{f_0}{2}\sqrt {\dfrac{1}{k_0m_1}}$.
 \item $\Rightarrow k_0 <\SI{1010}{N.m^{-1}}$.
 \item \begin{itemize}
\item Écart dynamique maximal : $\SI{5}{mm}$.
\item Écart de traînage (ou de vitesse) $\varepsilon_v$ en régime établi : $\SI{4}{mm}$.
\item Retard de traînage : $\SI{0,02}{s}$ -- cahier des charges validé -- Req 4.
\end{itemize}
\item .
\item $T_2 = \SI{0,1}{s}$.
\end{enumerate}
\end{solution}}
\fi